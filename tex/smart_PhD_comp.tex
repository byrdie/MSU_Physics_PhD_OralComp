\documentclass[10pt,letterpaper]{article}
\usepackage[latin1]{inputenc}
\usepackage[margin=1in]{geometry}
\usepackage{amsmath}
\usepackage{amsfonts}
\usepackage{amssymb}
\usepackage{graphicx}
\author{Roy Smart}
\title{Montana State University \\ Physics Department \\ PhD Oral Comprehensive Exam}
\begin{document}
	
	\maketitle
	
	\tableofcontents
	
	\section{Fresnel Diffraction}
	
		The general Kirchhoff integral in 2D is given as
		\begin{equation}
			\psi = C_0 \int_{-\infty}^{\infty} q(X) \frac{e^{i k (s_0 + s)}}{s_0 s} (\cos \theta_0 - \cos \theta) \; dX 
		\end{equation}
		where $C_0$ is some constant, $q(X)$ is the transmission function of the grating, $s_0$ and $s$ are distances from the point source $P'$ to the point $Q$ on the grating and from $Q$ to the test point $P$, respectively, and are defined by
		\begin{equation}
			\begin{split}
				&s_0 = \sqrt{X^2 + z_0^2} \\
				&s = \sqrt{(x-X)^2 + z^2} 
			\end{split}
		\end{equation}
		where we have taken the source point to be on the $z$-axis. If we use the small-angle approximation, where $z \gg x$, $s$ and $s_0$ become
		\begin{equation}
			\begin{split}
				&s_0 = z_0 \\
				&s = \sqrt{(x-X)^2 + z^2} 
			\end{split}
		\end{equation}
	
	\section{Self-imaging}
	\section{Gratings}
		\subsection{Small Ripples in Amplitude}
		\subsection{Small Ripples in Phase}
	
\end{document}